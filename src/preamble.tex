% !TEX root = proposal.tex

% For packages not directly related to MIT thesis class

%%%   Todo notes
\usepackage{xargs}                      % Use more than one optional parameter in a new commands
\usepackage[colorinlistoftodos,prependcaption,textsize=tiny]{todonotes}
\newcommandx{\unsure}[2][1=]{\todo[linecolor=red,backgroundcolor=red!25,bordercolor=red,#1]{#2}}
\newcommandx{\karen}[2][1=]{\todo[linecolor=blue,backgroundcolor=blue!25,bordercolor=blue,#1]{#2}}
\newcommandx{\info}[2][1=]{\todo[linecolor=OliveGreen,backgroundcolor=OliveGreen!25,bordercolor=OliveGreen,#1]{#2}}
\newcommandx{\improvement}[2][1=]{\todo[linecolor=greenc,backgroundcolor=greenc!25,bordercolor=greenc,#1]{#2}}
\newcommandx{\thiswillnotshow}[2][1=]{\todo[disable,#1]{#2}}
% ---------------------------------------------------------------------- %

%%%   Extra commands
%%    Partials
\newcommand{\PDer}[2]{%
    \frac{\partial #1}{\partial #2}
}
\newcommand{\PDerT}[2]{%
    \frac{\partial^2 #1}{\partial #2^2}
}
\newcommand{\PDerF}[2]{%
    \frac{\partial^4 #1}{\partial #2^4}
}
\newcommand{\PDerFM}[3]{%
    \frac{\partial^4 #1}{\partial #2^2 \partial #3^2}
}
\newcommand{\PDerTh}[2]{%
    \frac{\partial^3 #1}{\partial #2^3}
}
\newcommand{\PDerThM}[3]{%
    \frac{\partial^3 #1}{\partial #2 \partial #3}
}

\newcommand{\PDerM}[3]{%
    \frac{\partial^2 #1}{\partial #2 \partial #3}
}
\newcommand{\DerF}[2]{%
    \frac{\mathrm{d}^4 #1}{\mathrm{d} #2^4}
}
\newcommand{\Der}[2]{%
    \frac{\mathrm{d} #1}{\mathrm{d} #2}
}
\newcommand{\DerT}[2]{%
    \frac{\mathrm{d}^2 #1}{\mathrm{d} #2^2}
}
\newcommand{\DerTh}[2]{%
    \frac{\mathrm{d}^3 #1}{\mathrm{d} #2^3}
}

%% Extra math commands
\newcommand*{\rttensor}[1]{\overline{\overline{#1}}}
\newcommand{\bomega}{\textrm{\boldmath$\omega$}}
\DeclareMathOperator*{\argmin}{arg\,min}
\DeclareMathOperator{\Tr}{Tr}

%%% Tikz
%%  3D Plot
\makeatletter
\tikzoption{canvas is xy plane at z}[]{%
  \def\tikz@plane@origin{\pgfpointxyz{0}{0}{#1}}%
  \def\tikz@plane@x{\pgfpointxyz{1}{0}{#1}}%
  \def\tikz@plane@y{\pgfpointxyz{0}{1}{#1}}%
  \tikz@canvas@is@plane
}
\makeatother

% checkmarks etc.
\definecolor{darkgreen}{RGB}{26, 139, 26}
\newcommand{\Cross}{$\mathbin{\tikz [x=1.4ex,y=1.4ex,line width=.2ex, MITred] \draw (0,0) -- (1,1) (0,1) -- (1,0);}$}%
\newcommand{\Checkmark}{$\color{darkgreen}\checkmark$}

\newcommand{\todofig}{%
  \begin{tikzpicture}%
    \draw[line cap=rect,fill=MITred!50] (0,0) rectangle (\linewidth-\pgflinewidth,-5) node[pos=.5] {WORK IN PROGRESS};
  \end{tikzpicture}
}

%%% Nomenclature

% Copying \! to \negspace -- can't use \! in nomenclature
\let\negspace\!

\usepackage{nomencl}
\usepackage{etoolbox,ragged2e,mathtools} % Nomenclature

\DeclarePairedDelimiter{\abs}{\lvert}{\rvert}
\setlength{\nomitemsep}{-\parsep}
\addtolength{\nomitemsep}{4pt} % same as list of figures/tables

\renewcommand{\nomgroup}[1]{\medskip}

\renewcommand*{\nompreamble}{\markboth{\nomname}{\nomname}}
\renewcommand{\nomname}{Nomenclature}

\makenomenclature

%%% Equations

% cancel parts of equation
\usepackage{cancel}

\newcommand{\crit}{\textrm{crit}}
\newcommand{\fl}{\textrm{fl}}
\newcommand{\eng}{\textrm{eng}}
\newcommand{\smc}{\textrm{smc}}
\newcommand{\overbar}[1]{\mkern 1.5mu\overline{\mkern-1.5mu#1\mkern-0.75mu}\mkern 0.75mu}

% Transpose symbol
\newcommand{\trans}{\mathsf{\mathsmaller T}}

% To gray out zeros
\newcommand{\zz}{\textcolor{gray!75}{\mathbf{0}}}
\newcommand{\zzs}{\textcolor{gray!75}{0}}

\newcommand{\enn}{n}
\newcommand{\emm}{m}

%%% Acronyms
%%% Flutter
\newacronym{cfd}{cfd}{Computational Fluid Dynamics}
\newacronym{naca}{naca}{National Advisory Committee for Aeronautics}
\newacronym{fem}{fem}{Finite Element Method}


%%% Fonts

% special commands for lining figures
\newcommand{\SUTwo}{{\myliningnumfont SU2}}
\newcommand{\RAE}{{\myliningnumfont RAE2822}}
\newcommand{\NACASIX}{{\myliningnumfont NACA64A010}}
\newcommand{\TwoD}{{\myliningnumfont 2D}}
\newcommand{\ThreeD}{{\myliningnumfont 3D}}
\newcommand{\DTwo}{{\myliningnumfont D8.2}}
\newcommand{\DZero}{{\myliningnumfont D8.0}}
\newcommand{\NThree}{{\myliningnumfont N+3}}

% Hyphenation
\hyphenation{mecan-ique}
\hyphenation{Helm-holtz}

\usepackage{lipsum} % filler material
